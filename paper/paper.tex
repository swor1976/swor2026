%Aqui vai o comando para o tipo de documento e formatação da página
\documentclass[twocolumn,A4]{article}
% Pacotes que vamos (normalmente estão na maior parte dos documentos>
%%%%%%%%%%%%%%%%%%%%%%%%%%%%%%%%%%%%%%%%%%%%%%%%%%%%%%%%%%%%%%%%%%%%
%%%%%%%%%Para caracteres especiais
\usepackage[utf8]{inputenc}
\usepackage[TU]{fontenc}
%Para definir a lingua em que vamos escrever
\usepackage[brazil]{babel}
%Para gerar funções matemáticas
\usepackage{amsmath}
\usepackage{amssymb}
%Para inserir figuras
\usepackage{graphic}
%para links e metadados no PDF
\usepackage{hyperref}
%Para formatação melhorada
\usepackage{microtype}
%Para as citações
\usepackage[round,authoryear,sort]{natlib}

%configurações de formato do documento (chamam os pacotes importantes acima)
%Configura os metadados do nosso PDF
\hypersetup{
%links coloridos ao invés da caixa colorida
colorlinks,
%Cor dos links
allcolors=blue,
%título que aparece nos metadados
pdftitle={Mudança na temperatura média nos países nos últimos cincos anos},
%Autor que aparece nos metadados
pdfauthor={Leornado Uieda, Yago M. Castro, Athur Siqueira-Macedo},
%Permitir a quebra de linha no meio dos links
\breaklinks=true,
}

%Vamos começar EBA!
%%%%%%%%%%%%%%%%%%%%%%%%%%%%%%%%%%%%%%%%%%%%%%%%%%%%%%%%%%%%%%%%%%%%%%%%%%%%%%%%
\begin{document}
\title{Mudança na temperatura média de países nos últimos cincos anos}
\author{
	Leonardo Uieda\textsuperscript{1,2},
	Yago M.Castro\textsuperscript{1,2},
	Arthur Siqueira-Macedo\textsuperscript{1,2}
	\\[0.2cm]
	{\small
	\textsuperscript{1}Computer-Oriented Geoscience Lab
	\textsupperscript{2}Universidade de São Paulo, Brasil
	}
}
\date{\today}

\maketitle
\begin{abstract}
Analisando a série de dados históricos de temperatura de 225 países, foram abservados que os paises com a menor variação de temperatura terminam com ÃO
\end{abstract}

\end{document}

